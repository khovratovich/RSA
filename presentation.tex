
\documentclass[t]{beamer}

\usepackage{graphicx,comment,amssymb,hyperref,array}
\usetheme{Pittsburgh}
\usecolortheme{crane}
\usepackage{xcolor}
\author{Dmitry Khovratovich}
\institute{Ethereum Foundation}

\title{Factoring Integers}

\hypersetup{
    colorlinks=true,
    linkcolor=blue,
    filecolor=magenta,      
    urlcolor=blue,
    pdfpagemode=FullScreen,
}


\begin{document}    
\maketitle

\begin{frame}[c]
\begin{center}
    \Large{Problem of Factoring}
    \end{center}
\end{frame}

\begin{frame}{Problem of Factoring}
    Given $n$-bit number $N$, find prime $p_i$ and integer $d_i$ such that
    $$
    N = \prod_i p_i^{d_i}.
    $$
    The factorization is unique (up to the order of factors).\\[10pt]
    
    \pause Uniqueness of factorization is a property not only of the ring $\mathbb{Z}$ of integers, but also of $Z[x]$ and some other \emph{Dedekind domains}. %(for example, ring of integers over $Q[\sqrt{-7}]$. 
    
\end{frame}

\begin{frame}{Problem of Factoring}
    Given $n$-bit number $N$, find prime $p_i$ and integer $d_i$ such that
    $$
    N = \prod_i p_i^{d_i}.
    $$
   It is easy $(poly(n))$ to determine if $N$ is prime. Clearly, factoring is an NP problem, but it is not (known to be) NP-hard.
    
\end{frame}

\begin{frame}{Simple Factoring}
    For all $m<\sqrt{N}$
    \begin{itemize}
        \item Test if $m$ is prime;
        \item Test if $m$ divides $N$.
    \end{itemize}
    Complexity about $\sqrt{N}$.
%or
 %   $$
  %  2^{n/2} = e^{\frac{1}{2}ln N} =  L(1;0.5),
   % $$
    %where 
    %$$
    %L(\alpha;\beta) \approx e^{\beta ln^{\alpha}N}
    %$$
    
\end{frame}

\begin{frame}{Smarter Factoring}
\begin{enumerate}
    \item Generate $x,y$;
    \item Test if $x^2 = y^2 \pmod{N}$;
    \item If yes, $(x-y)(x+y)=0\pmod{N}$, so  $GCD(x-y,N)$ probably gives a factor of $N$, otherwise go to step 1.
    
\end{enumerate}
    
    Complexity about $\sqrt{N}$ again.
\end{frame}

\begin{frame}{Smarter Factoring}
\begin{enumerate}
    \item Generate $x,y$;
    \item Test if $x^2 = y^2 \pmod{N}$;
    \item If yes, $(x-y)$ probably divides $N$, otherwise go to step 1.
    
\end{enumerate}
How to improve?
\begin{itemize}
    \item Generate $x,y$ such that the equation is more likely to hold. 
\end{itemize}

\end{frame}

\begin{frame}[c]
\begin{center}
    \Large{Advance factoring methods}
    \end{center}
\end{frame}

\begin{frame}{Smart factoring}
Suppose for some $x_i >\sqrt{N}$ 
$$
x_i^2 =z_i =  2^{a_i}3^{b_i}5^{c_i} \pmod{N}
$$
 i.e. its factors are only 2,3,5.
Also
\begin{align*}
x_j^2 =z_j =  2^{a_j}3^{b_j}5^{c_j} \pmod{N}\\
x_k^2 =z_k =  2^{a_k}3^{b_k}5^{c_k} \pmod{N}\\
x_l^2 =z_l =  2^{a_l}3^{b_l}5^{c_l} \pmod{N}\end{align*}
The matrix 
$$
\begin{pmatrix}
a_i & b_i & c_i\\
a_j & b_j & c_j\\
a_k & b_k & c_k\\
a_l & b_l & c_l\\
\end{pmatrix}
$$
admits a linear combination $\bmod{\,2}$, so that
$\alpha(a_i,b_i,c_i) + \beta(a_j,b_j,c_j)+\gamma (a_k,b_k,c_k) + \delta(a_l,b_l,c_l)=(0,0,0) \pmod{2}$.

\end{frame}


\begin{frame}{Smart factoring}
\begin{align*}x_i^2 =z_i =  2^{a_i}3^{b_i}5^{c_i} \pmod{N}\\
x_j^2 =z_j =  2^{a_j}3^{b_j}5^{c_j} \pmod{N}\\
x_k^2 =z_k =  2^{a_k}3^{b_k}5^{c_k} \pmod{N}\\
x_l^2 =z_l =  2^{a_l}3^{b_l}5^{c_l} \pmod{N}\end{align*}
$$
\alpha(a_i,b_i,c_i) + \beta(a_j,b_j,c_j)+\gamma (a_k,b_k,c_k) + \delta(a_l,b_l,c_l)=(0,0,0) \pmod{2}.
$$
Therefore 
$$
v^2 = (x_i^{\alpha}x_j^{\beta}x_k^{\gamma}x_l^{\delta})^2 = 
4^{u_2}9^{u_3}25^{u_5} = w^2 \pmod{N}
$$
\pause This method works but the fraction of numbers divisible only by $2,3,5$ is low...

\end{frame}

\begin{frame}{Example}
    Factor number 143 this way:
    \begin{align*}
        14^2, 15^2, 16^2 &= fail\\
        17^2 = 289 &= 3\pmod{143}\\
        18^2 &=fail\\
        19^2 = 361 &= 75\pmod{143} = 3\cdot 5^2\\
        20^2 &=fail\\
        21^2 =441 &12\pmod{143} = 2\cdot 3^2\\
        22^2, 25^2, 26^2, \ldots, 33^2 &=fail\\
        34^2 =1156&=12\pmod{143} = 2^2\cdot 3
    \end{align*}
  \pause  Therefore
    $$
   37^2 =  (17\cdot 19)^2 = 3^2 \cdot 5^2 \pmod{143}
    $$
    and 
    $$
    (37-15)\cdot(37+15) = 0\pmod{143}; \quad GCD(22,143) = 11.
    $$
\end{frame}

\begin{frame}{Even smarter factoring}
Call $z$ $B$-smooth if all its factors smaller than $B$.
\begin{enumerate}
    \item Generate $x$;
    \item Test if $z=x^2 \pmod{N}$ is $B$-smooth. If yes, add it to the table. This is \textbf{sieving}.
    \item With $O(B)$ such $x$, find a linear dependency so that
    $$
    x_{i_1}^2x_{i_2}^2\cdots x_{i_B}^2 = w^2.
    $$
    This is \textbf{matrix step}.
\end{enumerate}
\pause Complexity:
\begin{itemize}
    \item The matrix step needs $O(B)$ combinations and works in $O(B^2)$ time (block Wiedemann algorithm for sparse matrices).
    \item The sieving step needs time 
    $$
    T\approx \frac{B\cdot C_{test}[z\text{ is smooth}]}{Pr_{z,N,B}[z\text{ is $B$-smooth}]}
    $$
\end{itemize}

The tradeoff is when the smoothness probability is $O(1/B)$, then both sieving and matrix step cost  $B^2$.
     
    

\end{frame}

\begin{frame}{Smoothness probability}


Smaller $B$: more trials but smaller matrux:
\begin{itemize}
    \item If $B=const$ then the probability that $x<N$ is $B$-smooth is $1/N^{\ln B}$.
    %L[1;1]   L[0;ln B] =  L[1;-1(1)/ln B] = N^{ln B}
    \item If $B = e^{\sqrt{ln N}}$ the probability is $1/\sqrt{B}$. Then the complexity to find $B$ numbers is $B^{3/2}$. Recall the matrix step is $B^2$.
    %L[1;1]  L[1/2;1]   = L[1/2;-1/2]
\end{itemize}

\end{frame}

\begin{frame}{Quadratic Sieve}

Quadratic sieve (Pomerance 1982):
\begin{enumerate}
    \item Let $x = \sqrt{N}+\epsilon$ where $\epsilon =o(\sqrt{N})$;
    \item Then $z=x^2 \approx 2\epsilon\sqrt{N} \pmod{N}$.
    \item Test if $z=x^2 \pmod{N}$ is $B$-smooth. \item After finding $O(B/\ln(B))$ such $x$, construct
    $$
    y^2 = z^2\pmod{N}.
    $$
\end{enumerate}
So we test smoothness for vector $\mathbf{z} = Q(\mathbf{x}) = \mathbf{x}^2-N$.
$$
Q(\lceil\sqrt{N}\rceil+1),Q(\lceil\sqrt{N}\rceil+2),\ldots,Q(\lceil\sqrt{N}\rceil+B^2)
$$
\end{frame}

\begin{frame}{Quadratic Sieve}
 Apparently $B$-smoothness can be tested faster than $O(B)$ per element for a big group of $x$:
\begin{itemize}
    \item Let $q <B$ be prime.
    \item Find $x\approx \sqrt{N}$ such that $Q(x)=x^2-N=0\pmod{q} $.
    \item Then for any $k$ we have
    $$
    Q(x+kq) = Q(x) = 0\pmod{q}
    $$
    i.e. $Q(x+kq)$ divisible by $q$.
    \item Repeat for all $q$ to get smooth $z$ in the list.
    \item This works for any polynomial $Q$.
\end{itemize}

Complexity is minimum for $B = e^{\sqrt{0.5 \ln N}}$
\end{frame}

\begin{frame}[c]
\begin{center}
    \Large{Number Field Sieve}
    \end{center}
\end{frame}

\begin{frame}[c]{Number field sieve}
    In Number Field Sieve (Pollard et al., 1993)  we  compute $z$ which are much smaller than $\sqrt{N}$ so that the smoothness probability is greater.
\end{frame}

\begin{frame}{NFS}
    \begin{itemize}
        \item Let $f$ be poly of degree $d$:
    $$
    f(m) = 0 \pmod{N}
    $$
    $m$ can be from $m$-ary representation of $N$.
   \pause \item  Let $\alpha$ be its root over $\mathbb{N}$. Then $\phi(h\in \mathbb{Q}[\alpha]) =h(m)\in Z_N$ is homomorphism.
    
    
\pause    \item If $\prod_i (a_i-b_i\alpha) = \beta^2\in \mathbb{Q}[\alpha]$ and $\prod_i(a-bm)=y^2$ then
    \begin{multline*}
    x^2 = (\phi(\beta))^2= \phi(\beta^2) = \phi(\prod_i (a_i-b_i\alpha)) = \\
    \prod_i \phi((a_i-b_i\alpha)) = 
    \prod_i (a_i-b_im) = y^2.
    \end{multline*}
\item We search for $(a,b)$ such that both $(a-bm)$  and $(a-\alpha m)$ are smooth, then there is a product that yields a square.
 \end{itemize}
   But how to find $(a,b)$ such that $(a-b\alpha)$ is smooth in $\mathbb{Q}[\alpha]$? %And what does this mean?

\end{frame}

\begin{frame}{Norms}
First, $Z[\alpha]$ is (almost) a unique factorization domain: any ideal can be written as a product of prime ideals.

    Define $\mathcal{N}(a-b\alpha) = b^df(a/b)$.
    \begin{itemize}
        \item It is multiplicative, so if $(a-b\alpha) $ is a square, so is $ b^df(a/b)$.
        \item Let us search for $(a,b)$ such that $b^df(a/b)$ is $B$-smooth using \emph{lattice sieve} (a variant of quadratic sieve). 
        \item We then solve   $\prod_i(a-b\alpha) = \beta^2$, i.e. compute square roots in $\mathbb{Z}[\alpha]$.
        \item[*] To make square roots possible we have to add extra information to $b^df(a/b)$ thus making more columns in matrix.
    \end{itemize}
\end{frame}

\begin{frame}{Summary of NFS}
    \begin{itemize}
        \item Define $f$ of degree $d$ (usually 5) with root $m$ in $\mathbb{Z}_N$;
        \item Find $B$ such $(a,b)$ such that both $(a-bm)$ and $b^df(a/b)$ are $B$-smooth;
        \item Using fast matrix algorithms, find
        $$
        \mathcal{I}:\quad \prod_{i \in \mathcal{I}}(a_i-b_im) = y^2
        $$
        \item By construction, $\prod_{i \in \mathcal{I}}(a_i-b_i\alpha) = \beta^2$ for some unknown $\beta$, it is solved by computing a square root in $Q[\alpha]$ (also a difficult problem).
        
     \end{itemize}
     The complexity is smaller since we need $a,b$ of size $B$, so that  $$
     (a-bm)<b^df(a/b) <B^{d}N^{1/d}\approx N^{0.4},
     $$
     so that both are much smaller than $\sqrt{N}= e^{0.5 \ln N}$.
     \end{frame}

\begin{frame}{Complexity}

     Complexity is
     $$
     L(n) = 2^{2.4\sqrt[3]{n}\sqrt[1.5]{\ln n}-16}
     $$
     which is about $50L(n)$ CPU cycles
     and smoothness bound
     $$
     B = \sqrt{L(n)} = 2^{1.2\sqrt[3]{n}\sqrt[1.5]{\ln n}-8}
     $$
     Current factorization records:
     \begin{itemize}
         \item 768-bit (2010): 2200 core-years;
         \item 795-bit (2019): 900 core-years;
         \item 829-bit (2020):  2900 core-years.
     \end{itemize}
     Translating into electricity costs:
     \begin{equation}\label{c795}
C(795) = 
\underbrace{2^{9.8}}_{\text{core-years}}\cdot
\underbrace{2^{3.1}}_{\text{year/kilo-hour}}\cdot
\underbrace{2^{5}}_{\text{Watt/core}}\cdot
\underbrace{2^{-5}}_{\text{USD/kWh}}=
2^{12.9}    
\end{equation}
 USD worth of electricity.
\end{frame}

\begin{frame}[c]
\begin{center}
    \Large{NFS Perspectives}
    \end{center}
\end{frame}

\begin{frame}{Complexity Perspectives}
For the smoothness bound $B$
    \begin{itemize}
        \item Sieving: test  $O(B^2)$ integers   for $B$-smoothness. $O(B)$ memory and $O(B^2)$ time. Not random memory access, can be parallelized. The memory is used predictably. 
        \item Linear: find a kernel element for a matrix of size $B\times B $ with $O(B)$ non-zero elements. Currently $O(B^2)$ time and $O(B)$ memory, but have potential $O(B^{1.5})$ time (with maybe more random memory access).
    \end{itemize}   
   \pause If we could run matrix step in $B^{1.5}$ time, we could increase the smoothness bound and break 1.25-longer numbers.
    
\end{frame}


\begin{frame}{Cost Estimates}
    \begin{itemize}
        \item \textbf{Optimistic}: CPUs  and GPUs remain the most efficient devices for factoring.
        \item \textbf{Realistic}: ASIC would reduce the costs by $2^7$ due to high memory requirements.
        \item \textbf{Conservative}: ASIC would bring the advantage of $2^{10}$ over clusters that we observe for memoryless algorithms. Minor advances in algorithms.
    \end{itemize}
\end{frame}

\begin{frame}{128-bit Security}
    Based on Bitcoin mining, we estimate that 80-bit AES key can be found on ASIC for $2^{19}$ USD worth of electricity.
    \begin{itemize}
        \item 128-bit security -- cost of 128-bit key recovery on ASIC ($2^{67}$ USD).
        \item 256-bit security -- should become 128-bit secure if a quadratic speed-up to algorithms is found.
    \end{itemize}
\end{frame}

\begin{frame}{Bounds for Moduli}
    \begin{table}[t]
    \centering
    \begin{tabular}{|c|c|c|c|c|c|}
        \hline 
        Security level &80 & 100 & 112 & 128 & 256   \\
        \hline\hline
        \multicolumn{6}{|c|}{Our estimates}\\\hline\
        Super-optimistic & 950 & 1600 & 2075 & 2850 &15500\\
        Optimistic& 1150 & 1850 & 2400 & 3225 & 16500\\
        Realistic & 1450 &2200 & 2800& 3700 & 17500\\
        Conservative & 1500 & 2350&2950& 3900 & 18500\\
        Super-conservative& 1800 & 2800 &3550& 4700 & 22200\\         \hline\hline
     
    \end{tabular}
    \caption{Security of RSA moduli.}
    \label{tab:sec}
\end{table}
\end{frame}


\begin{frame}[c]
\begin{center}
    \Large{Extras}
    \end{center}
\end{frame}

\begin{frame}{Elliptic curve factoring}
\begin{enumerate}
    \item Generate elliptic curve $\mathcal{E}$ over $Z_N$;
    \item Select random point $P$;
    \item Compute $P_2=[2]P$, then $P_3=[3][2]P$, then $P_4=[4][3][2]P$, etc.. Use GCD for inversion modulo $N$.
    \item Stop after $B$ primes or if GCD fails (=computes factor of $N$).
\end{enumerate}
Why it works?
\begin{itemize}
    \item Let $p$ be a factor, consider the same curve over $Z_p$ with order $N_p$.
    \item If $N_p$ is $B$-smooth, then $P_B = [k][N_p]P = \infty$ on this curve.
    \item If so the GCD calculation on the original curve should fail here.
\end{itemize}
        Works in $e^{\sqrt{2\ln p\ln\ln p}}$ time, so great for small factors and memoryless.
\end{frame}

\begin{frame}{Discrete logarithm}
  Prime field $F_p$, find $\log_g h$:
    \begin{enumerate}
    \item Generate $\chi$;
    \item Test if $z=g^{\chi} \pmod{p}$ is $B$-smooth: \textbf{sieving}.
    \item After finding $O(B/\ln(B))$ such $\chi$, solve a linear system $\pmod{p-1}$ and obtain
    $$
    \log_g {q} \;\forall q<B
    $$
    \item Try for different $\tau$ if
    $$
    h g^{\tau}
    $$
    is $B$-smooth. If so, you get $\log_g h$.
\end{enumerate}
Similar to NFS algorithms apply to prime fields and their extensions.
\end{frame}


\begin{frame}[c]
\begin{center}
    \Large{Questions?}
    \end{center}
\end{frame}

\end{document}