
\documentclass[a4paper]{article}
\usepackage[hmargin=2cm,vmargin=2cm]{geometry}


\pagestyle{plain}

\usepackage{amssymb,amsthm,amsmath,amsfonts,longtable, comment,array, ifpdf, hyperref,url}
\usepackage{graphicx}
\newtheorem{definition}{Definition}
\newtheorem{theorem}{Theorem}
\newtheorem{lemma}{Lemma}

\title{RSA Assumptions}
\author{Dmitry Khovratovich\\Ethereum Foundation}

\begin{document}

\maketitle

\section{Introduction}

Computing modulo an integer with secret factors, also known as RSA, is not only as a public-key cryptosystem, but also  a foundation of several other protocols: cryptographic accumulators~\cite{DBLP:conf/crypto/CamenischL02}, verifiable delay functions~\cite{DBLP:conf/eurocrypt/Wesolowski19,DBLP:conf/crypto/BonehBBF18}, polynomial commitment schemes~\cite{cryptoeprint:2019:1229}, range proofs, etc.. These protocols are proven secure (w.r.t. to certain adversaries) under various assumptions on the hardness of some problems in the RSA group.  Some of these assumptions had been shown to imply the other under certain conditions on the modulus or unconditionally, but many relations remain unknown. 

Only assumptions that have not been broken are listed here. However, the amount of public scrutiny from the cryptographic community into the research on their validity differs a lot. Whereas the RSA assumption has withstood the decades of attacks on the RSA encryption, the Adaptive Root assumption is relatively new so that finding it invalid would not surprise many scholars.

Several protocols that exploit the fact that the RSA group has unknown order for some parties has been recently carried out to other groups of unknown order~\cite{DBLP:conf/crypto/BonehBBF18,cryptoeprint:2019:1229}, particularly class groups~\cite{buchmann2001survey}. The latter is particularly interesting as finding the order of a class group is considered to be a hard problem without any secret setup like in RSA. Therefore, it is interesting whether various RSA assumptions also hold in other groups of unknown order.

Ethereum Foundation is particularly interested in the research on the Adaptive Root assumption, as it is necessary for the Wesolowski VDF protocol as well as for an efficient polynomial commitment protocol to hold~\cite{DBLP:conf/eurocrypt/Wesolowski19,cryptoeprint:2019:1229}. So far we found no evidence suggesting its validity or otherwise.

This report is a short survey on the existing RSA assumptions and relations among them. It does not cover assumptions arising in  groups of known order, such as various versions of Diffie-Hellman or Discrete Log. 

\section{Definitions}

Most assumptions are formulated with respect to the security parameter $\lambda$. This means that the group parameters are selected so that the assumption holds with overwhelming probability as a function of $\lambda$ (for example, with $1-\frac{1}{2^{\lambda}}$. The set of parameters as a function of $\lambda$ is modelled as a group generator $GGen(\lambda)$.

\begin{definition}
The \emph{RSA Assumption} states that no
efficient adversary can compute $l$-th roots a given random group element for a random $l$. Specifically,
it holds for $GGen$ if for any probabilistic polynomial time adversary $\mathcal{A}$:
$$
Pr
\begin{bmatrix}
&\mathbb{G}\xleftarrow{\$}GGen(\lambda)\\
u^l = w :
& (w,l)\xleftarrow{\$}\mathbb{G}\\
&u \xleftarrow{} \mathcal{A}(\mathbb{G},w,l)
\end{bmatrix}\leq \mathrm{negl}(\lambda)
$$
\end{definition}

\begin{definition}
The \emph{Strong RSA Assumption} states that no
efficient adversary can compute roots of a random group element. Specifically,
it holds for $GGen$ if for any probabilistic polynomial time adversary $\mathcal{A}$:
$$
Pr
\begin{bmatrix}
&\mathbb{G}\xleftarrow{\$}GGen(\lambda)\\
u^l = w,\; l>1 :
& w\xleftarrow{\$}\mathbb{G}\\
&(u,l) \xleftarrow{} \mathcal{A}(\mathbb{G},w)
\end{bmatrix}\leq \mathrm{negl}(\lambda)
$$
\end{definition}

Let $N$ denote the RSA modulus and $QR_N$ being the set of quadratic residues (those that are squares of other elements) in $Z_N = \{0,1,2,\ldots,N-1\}$.
\begin{definition}
The \emph{QR-Strong RSA Assumption} states that no 
efficient adversary can compute a root of a given random quadratic residue. Specifically,
it holds for $GGen$ if for any probabilistic polynomial time adversary $\mathcal{A}$:
$$
Pr
\begin{bmatrix}
&Z_N\xleftarrow{\$}GGen(\lambda)\\
u^l = w,\; l>1 :
& w\xleftarrow{\$}QR_N\\
&(u,l) \xleftarrow{} \mathcal{A}(Z_N,w)
\end{bmatrix}\leq \mathrm{negl}(\lambda)
$$
\end{definition}

\begin{definition}
The \emph{$r$-Strong RSA Assumption} states that an 
efficient adversary can compute at most $r$-th roots of a given random group element. Specifically,
it holds for $GGen$ if for any probabilistic polynomial time adversary $\mathcal{A}$:
$$
Pr
\begin{bmatrix}
&\mathbb{G}\xleftarrow{\$}GGen(\lambda)\\
u^l = w,\; l\neq r^k,\;k\in\mathbb{N} :
& w\xleftarrow{\$}\mathbb{G}\\
&(u,l) \xleftarrow{} \mathcal{A}(\mathbb{G},w)
\end{bmatrix}\leq \mathrm{negl}(\lambda)
$$
\end{definition}
Remarks:
\begin{itemize}
    \item For $r = 1$ the definition is identical to the standard Strong RSA Assumption. 
    \item For $r = 2$, the adversary
is efficiently able to take square roots. In class groups of imaginary quadratic order taking
square roots is easy~\cite{cryptoeprint:2019:1229}.
    \item In $r$-th order class groups taking $r$-th roots is easy~\cite{cryptoeprint:2019:1229}.
    \end{itemize}

\begin{definition}
The \emph{Adaptive Root Assumption}   holds for
$GGen$ if there is no efficient adversary $(\mathcal{A}_0, \mathcal{A}_1)$ that succeeds in the following task. First,
$\mathcal{A}_0$  outputs an element $w\in \mathbb{G}$ and some state $st$. Then, a random prime in $\mathrm{Primes}(\lambda)$ is chosen
and $\mathcal{A}_1(w,l,st)$ outputs $w^{1/l}\in\mathbb{G}$. For all efficient $(\mathcal{A}_0, \mathcal{A}_1)$:
$$
Pr
\begin{bmatrix}
&\mathbb{G}\xleftarrow{\$}GGen(\lambda)\\
& (w,st)\xleftarrow{}\mathcal{A}_0(\mathbb{G})\\
u^l = w\neq 1 :& l\xleftarrow{\$}\Pi_{\lambda}=\mathrm{Primes}(\lambda)\\
&u \xleftarrow{} \mathcal{A}_1(w,l,st)
\end{bmatrix}\leq \mathrm{negl}(\lambda)
$$
\end{definition}
Remarks:
\begin{itemize}
    \item In~\cite{cryptoeprint:2019:1229} $\mathcal{A}_0$ selects $w,st$ randomly, which is probably an error. Also $\mathcal{A}_1$ does not take $w$ as input, which does not make sense.
    \item The number of primes in $\Pi_{\lambda}$ should be exponential in $\lambda$: it is possible to precompute $w$ using $2^{|\Pi_{\lambda}|}$ exponentiations. Then, an adversary with $2^M$ memory can store intermediate exponents and compute adaptive roots using $2^{|\Pi_{\lambda}|-M}$ exponentiations for each.
\end{itemize}
\begin{definition}
The \emph{Order assumption}. For any probabilistic polynomial time adversary $\mathcal{A}$ computing the order of a random element is hard: 
$$
Pr
\begin{bmatrix}
&\mathbb{G}\xleftarrow{\$}GGen(\lambda)\\
u^l = 1 : &u \xleftarrow{\$}\mathbb{G}\\ 
& l \xleftarrow{} \mathcal{A}(\mathbb{G})\\
\end{bmatrix}\leq \mathrm{negl}(\lambda)
$$

\end{definition}

\begin{definition}
The \emph{Low Order assumption}. For any probabilistic polynomial time adversary $\mathcal{A}$ finding any element of low order is hard: 
$$
Pr
\begin{bmatrix}
&\mathbb{G}\xleftarrow{\$}GGen(\lambda)\\
u^l = 1,\;u\neq 1 :
& (u,l) \xleftarrow{} \mathcal{A}(\mathbb{G})\\
& \text{and }l<2^{poly(\lambda)}
\end{bmatrix}\leq \mathrm{negl}(\lambda)
$$

\end{definition}

\begin{definition}
The \emph{Fractional Root assumption}. For any probabilistic polynomial time adversary $\mathcal{A}$ 
$$
Pr
\begin{bmatrix}
&\mathbb{G}\xleftarrow{\$}GGen(\lambda)\\
u^l = g^{a} : & g \xleftarrow{\$}\mathbb{G}\\
& (u,l,a) \xleftarrow{} \mathcal{A}(\mathbb{G},g)\\
& \text{and }a,l<2^{poly(\lambda)}
\end{bmatrix}\leq \mathrm{negl}(\lambda)
$$

\end{definition}

\begin{definition}
The \emph{Diffie-Hellman Assumption} holds for
$GGen$ if no efficient $\mathcal{A}$ can compute $g^{ef}$ from $g,g^e,g^f$ for random $g,e,f$:
$$
Pr
\begin{bmatrix}
&\mathbb{G}\xleftarrow{\$}GGen(\lambda)\\
& (g,e,f)\xleftarrow{\$}\mathbb{G}\\
u = g^{ef} :& u \xleftarrow{} \mathcal{A}(g,g^e,g^f)
\end{bmatrix}\leq \mathrm{negl}(\lambda)
$$
\end{definition}

\begin{definition}
\emph{Discrete Logarithm} assumption 
holds for
$GGen$ if for all efficient $\mathcal{A}$:
$$
Pr
\begin{bmatrix}
&\mathbb{G}\xleftarrow{\$}GGen(\lambda)\\
& (u,w)\xleftarrow{\$}\mathbb{G}\\
w = u^l :& l \xleftarrow{} \mathcal{A}(u,w)
\end{bmatrix}\leq \mathrm{negl}(\lambda)
$$
\end{definition}

\begin{definition}
The \emph{Factoring} assumption states that for random safe primes $p,q$ it is difficult to factor $N=pq$.
\end{definition}


\section{Reductions and security}

\subsection{Trivial reductions}

\begin{enumerate}
    \item The Adaptive Root assumption implies the Low Order assumption. Indeed, for an element $w$ of order $l$ one can compute a $q$-th root by setting $u = w^{q^{-1}\bmod{l}}$. 
    \item The Strong RSA assumption implies the RSA assumption (trivially).
    \item The Strong RSA assumption implies the QR-Strong assumption (almost trivial, due to the size of $QR_N$).
    \item The Low Order assumption unconditionally holds in $QR_N$ because it contains no elements of low order.
    \item If the RSA modulus is the product of strong primes then the Order assumption in $QR_N$ is equivalent to factoring.
\end{enumerate}

\subsection{Nontrivial reductions}
\begin{theorem}\cite{bach1984discrete}
The Discrete Logarithm assumption in the RSA group is equivalent to the Factoring assumption.
\end{theorem}

\begin{theorem}\cite[Th. 1]{DBLP:conf/ccs/CramerS99}
The Strong RSA assumption is equivalent to the Fractional Root Assumption in the group of quadratic residues modulo $N$.
\end{theorem}

\subsection{Generic Group Model}

A generic group algorithm is a program that performs only group operations and equality checks. The group is modelled as an oracle $O$, who knows the group order $n$, and a random function $\sigma$ that maps $Z_N$ to bit strings, called the \emph{encoding}. The algorithm input is $\sigma(x_1),\sigma(x_2),\ldots,\sigma(x_k)$. The algorithm can query the oracle on pairs $(i,j)$, and the oracle returns $\sigma(x_i\pm x_j\pmod{n})$. Equivalently, it computes $\prod_{1\leq j \leq k}g_j^{a_j}$ and informs about equal elements in results.

It is crucial that a generic group algorithm does not have access to the internal representation of group elements, which are integers in RSA. 
Most RSA assumptions hold in the Generic Group Model. 
\begin{theorem}\cite{DBLP:conf/eurocrypt/DamgardK02}
The Strong RSA assumption holds in the Generic Group Model.
\end{theorem}
This implies that the RSA assumption is hard too. The Factoring assumption can not be formulated in the Generic Group Model as the group size is unknown to the algorithm.
\begin{theorem}\cite{cryptoeprint:2019:1229}
The Adaptive Root assumption holds in the Generic Group Model.
\end{theorem}
However, these results give little insight to the actual security of RSA assumptions, as most existing RSA attacks use the integer form of the group elements. For example, computing the Jacobi symbol (see below) in an RSA group is easy despite being provably hard in the Generic Group Model.

\subsection{Generic Ring Model}

Here we consider algorithms that are given the unit ring element $1$ and a single ring element $x$ as input and are supposed to output some element $y$. They can query the ring oracle using multiplication, division, and addition queries on the already known ring elements, and see if the oracle outputs a previously known element. Effectively these algorithms compute rational polynomial functions of $x$.

\begin{lemma}\cite{DBLP:conf/eurocrypt/AggarwalM09}\label{ref:lem1}
If there is a generic ring algorithm that computes $f(x)$ such that $f(x)\equiv 0 \bmod{n}$ on a non-negligible fraction of points then one can derive a factoring algorithm.
\end{lemma}

\begin{theorem}\cite{aggarwal2011equivalence}
If there is an generic ring algorithm that breaks the Strong RSA assumption by outputting rational functions $u=\frac{f(x)}{g(x)}$ and $l=\frac{h(x)}{q(x)}$, then $N$ can be factored with the same complexity.
\end{theorem}
The idea of the proof is the following. If $l$ is a rational function that is a constant $c$ on many points, we reduce $u^c$ to Lemma~\ref{ref:lem1}. Otherwise it must happen that $l$ as a rational function is a constant $c\bmod{q}$ which yields a factor by GCD.

These results hint that an algorithm that breaks an RSA assumption without breaking the Factoring assumption must use the integer representation of the RSA ring. An example is an algorithm for Jacobi symbol computation, which works in $O(\log^3(N))$. However, computing the Jacobi symbol (or to do any other membership test of certain algebraic property) is as hard as factoring in the Generic Ring Model~\cite{DBLP:journals/joc/JagerS13}.


\subsection{Pseudo-freeness}

Let $A$ be a set of constants and $\mathcal{F}(A)$ be the free group generated by $A$ i.e. the set of all finite products with multiples from $A$. 
\begin{definition}Let $X$ be a set of variables and consider equations of form $w_1 = w_2$ where $w_1,w_2\in\mathcal{A}\cup \mathcal{X}$, where $\mathcal{A}$ is a set of products of elements from $A$ and $\mathcal{X}$ is a a set of products of elements from $X$. A group $G$ is \emph{pseudo-free} if no efficient adversary can find an equation that does not have solutions in $\mathcal{F}(A)$ and a solution to this equation in $G$ (i.e. where $X$ and $A$ are mapped to some elements of $G$), where the mapping from $A$ to $G$ is a random function, chosen for every run of the adversary.
\end{definition}

Informally, a group is pseudo-free if no efficient algorithm can find a non-trivial relation among randomly chosen group elements.

\begin{theorem}\cite{DBLP:conf/tcc/Rivest04,DBLP:conf/eurocrypt/Micciancio05}
Assume that $N$ is the product of two safe primes. Then the Strong RSA assumption is equivalent to the RSA group being pseudo-free.
\end{theorem}

\begin{theorem}~\cite{DBLP:conf/tcc/Rivest04} The Order assumption holds in a pseudo-free group.
\end{theorem}

\begin{theorem}\cite{DBLP:journals/ijisec/HasegawaIST09}
The Diffie-Hellman assumption holds for a non-negligible fraction of bases $g$ in a pseudo-free group.
\end{theorem}

Therefore, the Strong RSA assumption implies the Order assumption if $N$ is the product of two safe primes. The situation when the Strong RSA assumption holds but the Adaptive Root assumption does not hold may thus only happen if the order of $w$ in the Adaptive Root assumption is unknown but roots are computable. 

\subsection{Non-equivalence}

One may ask if the listed assumptions can be proven equivalently hard for all groups. This is not possible.
\begin{theorem}\cite{LM71,DBLP:conf/tcc/Rivest04} The Order,  Strong RSA, and Discrete Log assumptions are independent: for each pair of assumptions there exist a group where only one of them holds.
\end{theorem}

\subsection{Common modulus}
It is easy to see that an Adaptive Root oracle is equivalent to an RSA decryption of the same ciphertext under multiple prime encryption exponents. It was shown that if the decryption exponent is relatively small ($N^{0.3}$ and smaller) then $N$ can be factored. It was also demonstrated that if multiple decryption exponents all smaller than $h(N)$ are present then the factorization can be obtained from public exponents, and $h(N)$ for which it is feasible, approaches $N$ if the number of exponents grows. It is unclear if this property can be used to reduce the Adaptive Root assumption to the Factoring assumption.


\begin{figure}
    \centering
    \includegraphics[scale=0.7]{pics/RSA-QR.pdf}
    \caption{Relations among different assumptions in the subgroup of quadratic residues $QR_N$ in the RSA group where $N$ is a product of safe primes. Equivalent properties are in the same box. Darker colours stand for stronger assumptions.}
    \label{fig:rsa}
\end{figure}

\begin{figure}
    \centering
    \includegraphics[scale=0.7]{pics/RSA-reg.pdf}
    \caption{Relations among different assumptions in an RSA subgroup with no restrictions on $N$. Except for factoring, the same structure holds for other groups.}
    \label{fig:rsa}
\end{figure}

\begin{figure}
    \centering
    \includegraphics[scale=0.8]{pics/GRA-assum.pdf}
    \caption{RSA assumptions in the Generic Ring Model}
    \label{fig:gra}
\end{figure}
\section{Open problems}

\begin{enumerate}
    \item How is the pseudo-freeness of the RSA group related to the Adaptive Root assumption? Does any property imply the other?
    \item Is the Adaptive Root assumption hard in the Generic Ring Model?
    \item Is the Adaptive Root assumption equivalent to factoring in the Generic Ring Model? 
    \item Can an AR oracle be used to compute the group order using the low-decryption-exponent attacks on RSA?
\end{enumerate}

\bibliographystyle{alpha}
\bibliography{secure.bib}

\end{document}
